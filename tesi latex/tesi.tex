%%%%%%%%%%%%%%%%%%%%%%%%%%%%%%%%%%%%%%%%%12pt: grandezza carattere
                                        %a4paper: formato a4
                                        %openright: apre i capitoli a destra
                                        %twoside: serve per fare un
                                        %   documento fronteretro
                                        %report: stile tesi (oppure book)
\documentclass[12pt,a4paper,openright,twoside]{report}
%
%%%%%%%%%%%%%%%%%%%%%%%%%%%%%%%%%%%%%%%%%libreria per scrivere in italiano
\usepackage[italian]{babel}
%
%%%%%%%%%%%%%%%%%%%%%%%%%%%%%%%%%%%%%%%%%libreria per accettare i caratteri
                                        %   digitati da tastiera come è à
                                        %   si può usare anche
                                        %   \usepackage[T1]{fontenc}
                                        %   però con questa libreria
                                        %   il tempo di compilazione
                                        %   aumenta
\usepackage[utf8]{inputenc}
%
%%%%%%%%%%%%%%%%%%%%%%%%%%%%%%%%%%%%%%%%%libreria per impostare il documento
\usepackage{fancyhdr}
%
%%%%%%%%%%%%%%%%%%%%%%%%%%%%%%%%%%%%%%%%%libreria per avere l'indentazione
%%%%%%%%%%%%%%%%%%%%%%%%%%%%%%%%%%%%%%%%%   all'inizio dei capitoli, ...
\usepackage{indentfirst}
%
%%%%%%%%%libreria per mostrare le etichette
%\usepackage{showkeys}
%
%%%%%%%%%%%%%%%%%%%%%%%%%%%%%%%%%%%%%%%%%libreria per inserire grafici
\usepackage{graphicx}
%
%%%%%%%%%%%%%%%%%%%%%%%%%%%%%%%%%%%%%%%%%libreria per utilizzare font
                                        %   particolari ad esempio
                                        %   \textsc{}
\usepackage{newlfont}
%
%%%%%%%%%%%%%%%%%%%%%%%%%%%%%%%%%%%%%%%%%librerie matematiche
\usepackage{amssymb}
\usepackage{amsmath}
\usepackage{latexsym}
\usepackage{amsthm}
\usepackage{cite}
\usepackage{listings}
\usepackage{hyperref} 
\usepackage[square,numbers,sort]{natbib} 
%\bibliographystyle{unsrt}
\bibliographystyle{unsrtnat}

\lstset{
	frame=single,
	breaklines=true
}


%
\oddsidemargin=30pt \evensidemargin=20pt%impostano i margini
\hyphenation{sil-la-ba-zio-ne pa-ren-te-si}%serve per la sillabazione: tra parentesi 
					   %vanno inserite come nell'esempio le parole 
%					   %che latex non riesce a tagliare nel modo giusto andando a capo.

%
%%%%%%%%%%%%%%%%%%%%%%%%%%%%%%%%%%%%%%%%%comandi per l'impostazione
                                        %   della pagina, vedi il manuale
                                        %   della libreria fancyhdr
                                        %   per ulteriori delucidazioni
\pagestyle{fancy}\addtolength{\headwidth}{20pt}
\renewcommand{\chaptermark}[1]{\markboth{\thechapter.\ #1}{}}
\renewcommand{\sectionmark}[1]{\markright{\thesection \ #1}{}}
\rhead[\fancyplain{}{\bfseries\leftmark}]{\fancyplain{}{\bfseries\thepage}}
\cfoot{}
%%%%%%%%%%%%%%%%%%%%%%%%%%%%%%%%%%%%%%%%%
\linespread{1.3}                        %comando per impostare l'interlinea
%%%%%%%%%%%%%%%%%%%%%%%%%%%%%%%%%%%%%%%%%definisce nuovi comandi
%

%%%%%%%%%%%%%%%%%%%%%%%%%%%%%%%%%%%%%%
% Comandi Custom %
%%%%%%%%%%%%%%%%%%%%%%%%%%%%%%%%%%%%%%
\newcommand{\xstudent}{Alessandro Pioggia}
\newcommand{\xsupervisor}{Giovanni Delnevo}

%%%%%%%%%%%%%%%%%%%%%%%%%%%%%%%%%%%%%%
% Fine Preambolo %
% Inizio documento%
%%%%%%%%%%%%%%%%%%%%%%%%%%%%%%%%%%%%%%


\begin{document}

	%%%%%%%%%%%%%%%%%%%%%%%%%%%%%%%%%%%%%%%%
	% Scelta delle dimensioni della pagina %
	%%%%%%%%%%%%%%%%%%%%%%%%%%%%%%%%%%%%%%%%

	%\setlength{\textwidth}{13.5cm}
	%\setlength{\textheight}{19cm}
	%\setlength{\footskip}{3cm}
	
	%%%%%%%%%%%%%%%%%%%%%%%%%
	% inizio prefazione
	%
	% pagina del titolo, indice, sommario
	%%%%%%%%%%%%%%%%%%%%%%%%%
	
	
%\textwidth=450pt
\oddsidemargin=25pt

\begin{titlepage}
\begin{center}
{{\Large{\textsc{Alma Mater Studiorum}}}\\
{\Large{\textsc{Universit\`a di Bologna}}} \\
{\textsc{Campus di Cesena}} \rule[0.1cm]{14cm}{0.1mm}
		\rule[0.5cm]{14cm}{0.6mm}
DIPARTIMENTO DI INFORMATICA – SCIENZA E INGEGNERIA
Corso di Laurea in Ingegneria e Scienze Informatiche }
\end{center}
\vspace{15mm}
\begin{center}
{\LARGE{\bf TRADUZIONI AUTOMATICHE}}\\
\end{center}
\vspace{40mm}
\par
\noindent
\begin{minipage}[t]{0.47\textwidth}
{\large{\bf Relatore:\\
Prof. \xsupervisor}}
\vspace{5mm}
\end{minipage}
\hfill
\begin{minipage}[t]{0.47\textwidth}\raggedleft
{\large{\bf Presentata da:\\
\xstudent}}
\end{minipage}
\vspace{20mm}
\begin{center}
{\large{\bf Sessione III\\%inserire il numero della sessione in cui ci si laurea
Anno Accademico 2021-2022}}%inserire l'anno accademico a cui si è iscritti
\end{center}
\end{titlepage}



	\begin{titlepage}                       %crea un ambiente libero da vincoli
                                        %   di margini e grandezza caratteri:
                                        %   si pu\`o modificare quello che si
                                        %   vuole, tanto fuori da questo
                                        %   ambiente tutto viene ristabilito
%
\thispagestyle{empty}                   %elimina il numero della pagina
\topmargin=6.5cm                        %imposta il margina superiore a 6.5cm
\raggedleft                             %incolonna la scrittura a destra
\large                                  %aumenta la grandezza del carattere
                                        %   a 14pt
\em                                     %emfatizza (corsivo) il carattere
Desidero dedicare la tesi a coloro che lo hanno reso possibile.
%
%%%%%%%%%%%%%%%%%%%%%%%%%%%%%%%%%%%%%%%%
\clearpage{\pagestyle{empty}\cleardoublepage}%non numera l'ultima pagina sinistra
\end{titlepage}
    
    %
%%%%%%%%%%%%%%%%%%%%%%%%%%%%%%%%%%%%%%%%
\pagenumbering{roman}                   %serve per mettere i numeri romani
\chapter*{Introduzione}                 %crea l'introduzione (un capitolo
                                        %   non numerato)
%%%%%%%%%%%%%%%%%%%%%%%%%%%%%%%%%%%%%%%%%imposta l'intestazione di pagina
\rhead[\fancyplain{}{\bfseries
INTRODUZIONE}]{\fancyplain{}{\bfseries\thepage}}
\lhead[\fancyplain{}{\bfseries\thepage}]{\fancyplain{}{\bfseries
INTRODUZIONE}}
%%%%%%%%%%%%%%%%%%%%%%%%%%%%%%%%%%%%%%%%%aggiunge la voce Introduzione
                                        %   nell'indice
\addcontentsline{toc}{chapter}{Introduzione}
Questa \`e l'introduzione.
%%%%%%%%%%%%%%%%%%%%%%%%%%%%%%%%%%%%%%%%%non numera l'ultima pagina sinistra
\clearpage{\pagestyle{empty}\cleardoublepage}

\tableofcontents                        %crea l'indice
%%%%%%%%%%%%%%%%%%%%%%%%%%%%%%%%%%%%%%%%%imposta l'intestazione di pagina
\rhead[\fancyplain{}{\bfseries\leftmark}]{\fancyplain{}{\bfseries\thepage}}
\lhead[\fancyplain{}{\bfseries\thepage}]{\fancyplain{}{\bfseries
INDICE}}
%%%%%%%%%%%%%%%%%%%%%%%%%%%%%%%%%%%%%%%%%non numera l'ultima pagina sinistra
\clearpage{\pagestyle{empty}\cleardoublepage}
\listoffigures                          %crea l'elenco delle figure
%%%%%%%%%%%%%%%%%%%%%%%%%%%%%%%%%%%%%%%%%non numera l'ultima pagina sinistra
\clearpage{\pagestyle{empty}\cleardoublepage}
\listoftables                           %crea l'elenco delle tabelle
%%%%%%%%%%%%%%%%%%%%%%%%%%%%%%%%%%%%%%%%%non numera l'ultima pagina sinistra
    
    

	
	%%%%%%%%%%%%%%%%%%%%%%%%%
	% inizio corpo del documento
	%
	% sequenze delle varie sezioni
	% è consigliato mantenere una struttura logica ben definita per separare le sezioni
	% si consiglia di reificare tale struttura fisicamente sul file system
	%%%%%%%%%%%%%%%%%%%%%%%%%
	

	
	% inclusione delle sezioni
	\clearpage{\pagestyle{empty}\cleardoublepage}
\chapter{Contesto}                %crea il capitolo
%%%%%%%%%%%%%%%%%%%%%%%%%%%%%%%%%%%%%%%%%imposta l'intestazione di pagina
\lhead[\fancyplain{}{\bfseries\thepage}]{\fancyplain{}{\bfseries\rightmark}}
\pagenumbering{arabic}                  %mette i numeri arabi


\section{Statistical machine translation - STM}

Gli STM, meglio conosciuti come statistical machine translators sono modelli di traduzione introdotti nel 1947, che con la successiva introduzione dei personal computers, gli scienziati sono stati in grado di applicare. Si tratta di modelli che, attraverso la teoria della probabilità, unita a concetti statistici, sono in grado di predirre un risultato (o target), a partire da una larga quantità di dati in ingresso (source). Il meccanismo introdotto è utilizzabile per:
\begin{itemize}
	\item tradurre del testo (non parole singole), la predittività è in grado di legare correttamente le parole fra loro, dandogli un senso;
	\item implementare delle funzioni di autocompletamento, ad esempio, a partire da una domanda, tenta di indovinare una possibile risposta.
\end{itemize}

\subsection{Large data set}

Dal momento che i modelli citati si basano principalmente sulla statistica, è necessario avere a disposizione una grande quantità di dati, catalogati ed organizzati in un data set.
In particolare il data set, è ottenuto ed estratto a partire da traduzioni effettuate in precedenza (effettuate ed approvate da esseri umani, non AI) e prende il nome di corpus.
Le organizzazioni mondiali più grandi mettono a disposizione le loro banche dati, alle quali si può accedere rispettando la privacy degli individui.  


\subsection{Teoria della probabilità}

La predittività che gli SMT è ottenuta sfruttando modelli probabilistici, ottenuti a partire dalle nozioni derivanti dalla teoria della probabilità, che tratta i fenomeni aleatori. Un fenomeno è un qualunque accadimento che porta ad un risultato, o esito, si dice deterministico se il risultato può essere calcolato o previsto con esattezza prima dell'effettuazione del fenomeno. Si dice invece aleatorio se non è possibile prevedere il risultato. L'insieme di tutti i possibili risultati di un fenomeno si indica con $\Omega$. L'evento è un sottoinsieme di $\Omega$, ovvero è l'insieme dei possibili risultati di un fenomeno aleatorio. Una valutazione di probabilità è invece, una funzione che associa ad ogni evento un numero tanto più grande quanto più riteniamo che tale evento possa accedere. Viene denotato da $P(E)$, per descrivere la valutazione dell'evento $E$. 
\\
Esempio dimostrativo:
\begin{itemize}
	\item $\Omega$ = $\{$ "okay, see you later", "oh really, I'm sorry", "don't bother me please!"$\}$
	\item $E$ = "la risposta estratta è "okay, see you later"";
	\item $P(E)$ = $\dfrac{1}{3}$ 
\end{itemize}
A partire da un insieme di possibili eventi $\Omega$
, la probabilità che la frase selezionata venga estratta è 1 su 3. 

\paragraph{Il fulcro} della predittività del modello ruota intorno alla probabilità condizionale, di conseguenza viene sfruttato il teorema di Bayes.
\subsubsection{Probabilità condizionale}

Considerando due eventi distinti $A$ e $B$, il calcolo della probabilità condizionale consta nel definire la probabilità che accada l'evento $B$ sapendo che è giù avvenuto $A$. Questa probabilità la definiamo come $P(A|B)$ e si calcola nel seguente modo: 
$P(B|A)$ = $\dfrac{A \cap B}{A}$, questo perché non è altro che la proporzione di $A \cap B$ in $A$.

\subsubsection{Formula di Bayes}

$P(A | B)$ = $\dfrac{ P(B | A) \cdot P(A)}{P(A)}$

\subsection{Variabili aleatorie continue}

Intuitivamente, le variabili casuali continue sono quelle che possono assumere un insieme continuo di valori, al contrario delle distribuzioni discrete, per le quali l'insieme dei possibili valori ha cardinalità al più numerabile. Inoltre, mentre per una distribuzione discreta un evento con probabilità zero è irrealizzabile (come, ad esempio, ottenere 3½ da un lancio di un dado tradizionale), questo non è vero nel caso di una variabile casuale continua. Ad esempio, misurando la lunghezza di una foglia di quercia, è possibile ottenere il risultato 3½ cm, ma questo ha probabilità zero poiché vi sono infiniti possibili valori tra 3 cm e 4 cm. Ognuno di questi ha probabilità zero, ma la probabilità che la lunghezza della foglia sia nell'intervallo (3 cm, 4 cm) è non nulla. Questo apparente paradosso è causato dal fatto che la probabilità che una variabile casuale X assuma valori in un insieme infinito, come un intervallo, non può essere calcolata semplicemente sommando la probabilità dei singoli valori.

\subsection{Phrase table}
In SMT, the MT engineer builds a Translation Model using the frequency of phrases appearing the training corpus into a table. This table stores the phrase and the number of times this repeats over the entirety of the training corpus. The more frequently a phrase is repeated in a training corpus, the more probable the target translation is correct. Each phrase (stored in the Phrase Table) can range from one to five words in length. This phrase table is referred to as the Translation Model.

Consequently, the MT engineer is using a probability model to hit on the right source target translation combination. The process is evolutionary as the corpus is refined and adjusted after each translation run to eliminate/adjust any anomalies. The more frequently the corpus is used, the more perfected it becomes. The development of the corpus quality is a continuous organic R and D process of a highly valuable translation asset.

Additionally, the MT engineer builds a secondary model using the target translation data. This model helps determine the order in which the engineer needs to assemble phrases (from the Phrase Table) in order to optimise translation Fluency i.e. to give the translated text its natural language flow. Fluency ensures that literal translations (i.e. the words are all there, but the sense of the sentence is not) are replaced by a more natural sounding translation.

\subsection{Modello probabilistico}

In questa sottosezione, mostrerò la costruzione di un possibile modello probabilistico, il quale ha lo scopo di predirre la parola mancante all'interno della frase.

\subsubsection{Dati}

\\
Frase: "Marco oggi ha <espressione mancante> il pallone"
\\
$E_{1}$ = "l'espressione mancante è un sostantivo"
\\
$E_{2}$ = "l'espressione mancante è un verbo"


\newpage

\section{Neural machine translation - NMT}                 %crea la sezione
La neural machine translation (in italiano traduzione automatica neurale (NMT))  è un approccio alla traduzione automatica che utilizza una rete neurale per prevedere la probabilità di una sequenza di parole, modellando in genere intere frasi in un unico modello integrato. La creazione del modello sfrutta interamente il machine learning, dunque è in grado di apprendere autonomamente sulla base dei training alla quale prende parte. 

\subsubsection{Differenze dai modelli tradizionali}

\begin{itemize}
	\item Rispetto ai modelli di traduzione tradizionali, chiamati SMT(statical machine translation), sfruttano una quantità infinitesimale di memoria;
	\item grazie all'utilizzo del deep learning nella creazione della rete neurale, è possibile optare per un addestramento del tipo end-to-end, questo permette di semplificare la pipeline di sviluppo, con conseguente aumento di prestazioni non indifferente. Per andare più nello specifico:
	\begin{itemize}
		\item una pipeline tradizionale ha questo aspetto: 
		\\
		voce(input) $\rightarrow$ riduzione del rumore (noise)   $\rightarrow$ estrazione dei fonemi $\rightarrow$ composizione delle parole $\rightarrow$ trascrizione
		\item una pipeline che sfrutta il deep learning si presenta invece così:
		\\
		voce(input) $\rightarrow$ $DNN$  $\rightarrow$ trascrizione	 
	\end{itemize}
	Osservando le pipeline proposte, è possibile osservare che l'approccio tradizionale porta con sè una complessità non indifferente, ogni singolo layer citato deve essere ottimizzato separatamente, attraverso un preciso criterio. La pipeline creata a partire da una rete neurale invece, semplifica la comunicazione fra i due agenti(input $\rightarrow$ output desiderato).
	\\ 
	Nota: DNN = deep neural network
\end{itemize}

\subsection{RNN}

Prima di descrivere in maniera analitica e precisa la struttura che permette la costruzione di un modello di encoding-decoding è necessario definire le RNN, o Recurrent Neural Networks.
Le RNN sono delle reti neurali molto interessanti, perché a differenza dalle tradizionali straight forward neural networks, in cui l'informazione può andare solo in un verso (in avanti), in questo tipo di reti i nodi possono essere interconnessi ai livelli precedenti, ammettendo dei loop. Il fatto che i neuroni (o nodi, intesi come elementi che compongono la rete neurale) possano propagare il segnale in tutte le direzioni, permette di introdurre intrinsecamente il concetto di memoria. Questo perché, l'output di un neurone può ipoteticamente influenzare sè stesso in un diverso istante temporale (rispetto al presente). In particolare, una RNN richiede, oltre all'input tradizionale, lo hidden state, che non è altro che l'output prodotto dallo stesso neurone al passo precedente.

\textbf{Perché sfruttare le RNN?}
Le RNN, grazie alla memoria, possono operare su dati dinamici mentre le SNN (straight forward neural networks) operano unicamente su dati statici. Grazie a questa proprietà, le reti neurali ricorrenti, sono in grado di trattare delle sequenze temporali, variabili nel tempo. Un esempio di applicazione può convenire nella interpretazione del linguaggio dei segni. In questo ambito applicativo non è richiesto solo di riconoscere la mano, bensì di comprendere la sequenza di movimenti per poter dedurre ed interpretare il significato dei gesti. Quest'ultima constatazione ci rende in grado di percepire l'importanza di questo tipo di rete neurale nell'ambito delle traduzioni automatiche, in quanto si ha a che fare con sequenze di parole, non necessariamente definite in maniera statica che, per dare un senso logico alla frase, necessitano un meccanismo che sfrutti la memoria (e non solo). Per concludere l'approfondimento, considero fondamentale precisare che, by default, la memoria di un neurone è piuttosto breve, dunque gli output generati dai primi layer, difficilmente condizioneranno quelli presenti in livelli più avanzati. A questo proposito sono state studiate delle reti neurali a lunga memoria, quali le 
LSTM (Long Short Term Memory) e le GRU (Gated Recurrent Units).

\subsection{Transformers}

Nell'ambito del NMT, il modello maggiormente utilizzato è il transformer, che negli ultimi anni ha preso il posto delle reti neurali RNN, descritte nella sezione precedente. 

\subsubsection{Caratteristiche e peculiarità}

La tipologia di modelli citati in questa sezione sono nati con l'obiettivo di evitare la ricorsione, facendo spazio a delle tecnologie che garantissero un tipo di computazione che ammettesse il parallelismo. Le caratteristiche principali sono le seguenti:
\begin{itemize}
	\item data una sequenza di dati in input, a differenza delle reti RNN, che operano word-by-word ( ossia hanno la necessità di analizzare ogni singolo elemento della struttura), il transformers sono in grado di processare l'input \textbf{"as a whole"}, ossia come un unico grande dato;
	\item l'introduzione della \textbf{self-attention}, che consente un notevole miglioramento nella predittività, ha dato il nome all'articolo che ha presentato questa tipologia di modelli, ovvero "Attention is all you need";
	\item il \textbf{positional embedding}: si tratta di una tecnica che, insieme all'attention, ha come obiettivo quello di eliminare i processi che sfruttano la ricorsione. In particolare, l'idea è di assegnare dei pesi legati alla posizione di una parola all'interno della frase, in modo da verificare quali siano le tuple di parole che hanno più probabilità di essere accostate.
\end{itemize}

Il primo punto è fondamentale perché consente al modello di non dipendere da neuroni di layer lontani, processando l'intero input in una sola volta, viene eliminato il concetto di perdita di memoria. Il tutto può essere chiarito attraverso un semplice esempio:
\\
Consideriamo un ipotetico articolo, supponendo che i puntini rappresentino il testo di intermezzo: 
\\\\ \textit{La nazionale italiana di basket ....... Pozzecco ha deciso di intervenire in sala stampa, ringraziando lo staff}.\\

Dato questo input, il transformer, analizza l'articolo nel suo intero, mentre un modello tradizionale considera parola per parola, ovvero: $\{$0:La, 1:nazionale, 2: italiana, 3: di, 4: Basket, ..., 1200:Pozzecco $\}$. A giudicare da questo esempio si può osservare che è impossibile mantenere il contesto analizzando una parola per iterazione, specialmente se si ha necessità trovare un nesso fra due parole con indici molto distanti. 
 
\subsubsection{Descrizione tecnica dettagliata del modello}

Si tratta di un sistema di encoding-decoding. 
In linea generale, l'encoder mappa l'input ricevuto in un vettore continuo, che contiene tutte le informazioni apprese dai dati ricevuti in ingresso. Il decoder a partire dal vettore continuo, passo per passo, genera un unico output. Il singolo output generato è al contempo condizionato dagli ulteriori vettori continui dati in pasto al decoder in precedenza. Questo passaggio avviene ricorsivamente, fino a quando il decoder non incontra il token $<$end$>$, rappresentante la end of sentence (in italiano, fine della frase). 

\subsubsection{Input embedding}

Il primo passo è passare il nostro input al word embedded layer, il quale può essere considerato come una lookup table che contiene dei fattori di rappresentazione per ogni parola. In particolare, ogni parola viene mappata in un vettore avente valori continui, che la rappresentano, dal momento che il sistema apprende attraverso pattern numerici. 

\subsubsection{Positional encoding}

In seguito alla definizione dell'embedded layer è necessario passargli delle informazioni circa la collocazione di ogni vettore all'interno della rete neurale, dal momento che non è più possibile sfruttare la ricorsione a questo scopo. Per fare ciò è stata studiata una brillante soluzione, che verte sull'utilizzo di queste due formule:
\begin{itemize}
	\item $P(pos, 2i + 1) = \cos(\dfrac{pos}{10.000^{2i/dmodel}})$
	\item $P(pos, 2i) = \sin(\dfrac{pos}{10.000^{2i/dmodel}})$
\end{itemize}
Per ogni step di esecuzione di indice pari, genera un vettore di valori continui attraverso la funzione coseno. Per quelli di indice dispari invece, il vettore viene generato sfruttando il seno. Ad ogni vettore risultante viene sommato l'embedding vector corrispondente (ottenuto al primo step). Per la generazione si utilizzano le funzioni seno e coseno per via delle loro proprietà lineari, che l'intelligenza artificiale è in grado di apprendere con maggiore facilità. 

\subsubsection{Encoding layer}
Come introdotto inizialmente, l'encoder mappa l'input ricevuto in un vettore numerico contenente valori continui, rappresentanti le informazioni apprese dai dati ricevuti in ingresso. Contiene due sottomoduli:
\begin{itemize}
	\item il primo è una unità dedicata alla self- attention chiamata \textbf{multi head attention layer}. Andando nello specifico, ;
	\item l'altro è una vera e propria \textbf{rete neurale feed forward. }
\end{itemize}
I due layer sono collegati fra loro non direttamente, fra loro si interpone un layer intermedio che si occupa della normalizzazione dell'output. 

\subsubsection{Multi-head attention layer}

L'unità citata nel titolo, implementa uno specifico meccanismo di attention chiamato self-attention. Questa tecnologia consente ad un modello, attraverso principi combinatori accurati, di associare le parole presenti nell'input. Ad esempio, supponendo di avere la frase {"ciao", "come", "stai"}, l'unità potrebbe ipoteticamente correlare : {"ciao", "come"} oppure {"stai", "ciao"}. Attraverso questo tipo di associazioni, il modello può ad esempio interpretare la tipologia di frase, rendendosi conto in questo caso, di avere a che fare con una domanda. Queste intuizioni, permettono alla rete neurale di strutturare un possibile output.

\paragraph{Perché il meccanismo funzioni} 
l'input deve essere passato come parametro a tre distinti layer, fra loro connessi, i quali genereranno i seguenti vettori continui:
\begin{itemize}
	\item query vector;
	\item key vector;
	\item value vector.
\end{itemize}

In particolare il query vector contiene la richiesta, generalmente in formato json, che verrà etichettata dal key vector. Questa operazione è molto simile a quella che viene effettuata durante una richiesta ad un motore di ricerca, come ad esempio quello di youtube. Quando viene effettuata la ricerca, google etichetta il testo inserito dall'utente, in particolare associa la ricerca a descrizione, autore o titolo del video. Finita l'etichettatura, interroga il database, cercando i migliori match, ovvero i video strettamente compatibili con la richiesta.
\\
Il vettore delle query dunque, viene moltiplicato per il vettore delle chiavi (o key vector), ottenendo come risultato la matrice degli scores, o punteggi. Le celle della matrice risultante contengono un valore numerico, il quale indica quanto forte sia il collegamento fra una parola e l'altra, ovvero l'attenzione che occorre porre. Quindi, semplificando il concetto, all'interno di un arco temporale ogni parola avrà a disposizione un punteggio in relazione ad ogni altra parola presente. Il focus aumenta con l'aumentare del valore contenuto nella cella.
\begin{table}[h]                        %ambiente tabella
	%(serve per avere la legenda)
	\begin{center}                          %centra nella pagina la tabella
		\begin{tabular}{r|c|c}                  %tre colonne con righe verticali
			%   prodotte con |
			\hline \hline                           %inserisce due righe orizzontali
			$81$ & $18$ & $91$\\           %& separa le colonne e con
			\hline                                  %inserisce una riga orizzontale
			$509$ & $1000$ & $230$\\           %  \\ va a capo
			\hline                                  %inserisce una riga orizzontale
			$300$ & $200$ & $3300$\\
			\hline \hline                           %inserisce due righe orizzontali
		\end{tabular}
		\caption[legenda elenco tabelle]{Ipotetica matrice degli scores, per semplicità utilizzerò la sua nomenclatura originale, ovvero score matrix.}
	\end{center}
\end{table}

Osservando la matrice in formato tabellare sovrastante, si nota subito che l'ordine di grandezza del contenuto delle celle necessita una sorta di normalizzazione, per evitare che, la moltiplicazione con altre matrici, generi valori troppo grandi. A questo proposito si effettuano le seguenti operazioni:
\begin{itemize}
	\item $\dfrac{score matrix}{\sqrt{d_{k}}}$ $\rightarrow$ divido la matrice per la dimensione delle query e delle keys;
	\item $softmax(x)_{i} = \dfrac{exp(x_i)}{\sum_{j} exp(x_j)}$ $\rightarrow$ in questo passaggio, viene presa come input la matrice ottenuta al passo precedente, $x$ un elemento arbitrario. La matrice risultante chiamata sealed matrix, conterrà per ogni cella un valore probabilistico, il quale range va da $0$ ad $1$. Alle probabilità più alte verrà attribuita una maggiore attenzione (da qui, il termine attention), mentre quelle inferiori alla soglia verranno ignorate, perché non utili. This allows the model to be more confident on which words to attend to;
	\item $sealed matrix \cdot value matrix = output$ $\rightarrow$ moltiplico la matrice ottenuta al passaggio precedente, che può essere considerata una matrice di attention, per il vettore dei valori definito inizialmente. L'output, come già accennato, oscurerà le parole con un punteggio softmax più basso, dando la priorità alle altre.
	\item infine nell'ultimo passaggio, l'$output$ ottenuto al passaggio precedente viene passato in input ad un layer lineare, che sarà in grado di processarlo.
\end{itemize}

Perché si parli di multi-headed attention, è necessario che, i vettori presi originariamente in input (query, key e value vectors), vengano scissi $n$ volte e ogni tripletta di vettori ottenuta, attraversi lo stesso processo di self-attention individualmente. La scissione dei vari array, si basa su criteri prevalentemente dimensionali, in quanto, a partire dal  vettore originale, lungo $length$, si ricavano $n$ array, che rappresentano una divisione dello stesso in $n$ sottoarray.
Ogni singolo processo di self-attention prodotto prenderà il nome di head, da questo deriva il nome di multi-headed attention layer. 

\paragraph{Il seguente esempio} può chiarire il concetto riguardante la scissione:

\begin{itemize}
	\item array originali: $q=\{"hi", "how", "are", "you", "my", "friend"\}$, $k=\{1, 2, 3, 4, 5, 6\}$, $v=\{0.1, 0.2, 0.3, 0.4, 0.5, 0.6\}$ con singolo layer di self attention $h$, potrebbe diventare:
	\begin{itemize}
		\item $\{q_1 = \{"hi", how"\}, k_1=\{1, 2\}, v_1=\{0.1, 0.2\}\}$ $\rightarrow$ $head_1$
		\item $\{q_2 = \{"are", "you"\}, k_2=\{3, 4\}, v_2=\{0.3, 0.4,\}\}$ $\rightarrow$ $head_2$
		\item $\{q_3 = \{"my", "friend"\}, k_3=\{5, 6\}, v_3=\{0.5, 0.6\}\}$ $\rightarrow$ $head_3$
	\end{itemize}
	Abbiamo dunque ottenuto tre tuple di array, di cardinalità 3 e ad ogni tupla è associata una unità individuale di self-attention.
\end{itemize}

\newpage

\paragraph{Gli output} prodotti da ogni singolo processo derivante dalla scissione, al termine della operazione verranno concatenati in un unico oggetto, che sarà a tutti gli effetti l'input del layer lineare, introdotto nell'ultimo punto della sottosezione precedente. Il fatto di avere a disposizione un input generato da $n$ diversi processi di self-attention, fa in modo che, ogni head unit impari attraverso processi logici diversi e concatenandoli, si è in grado di offrire al modello un potere di rappresentazione più spiccato. 

\paragraph{Per riassumere} Il multi-headed-attention è un modulo, compreso in una rete di trasformatori,  che riceve in input i pesi relativi all'attention (o attention weights)  e produce un output, che contiene informazioni codificate. Le informazioni dicono al modello con quale frequenza e modalità le parole di una sequenza, devono essere associate.

\subsubsection{Residual connections}

Le residual connections (o connessioni residue), sono appartenenti ad un tipo di rete neurale chiamata \textbf{residual neural networks}. Si tratta di una rete la quale, attraverso le proprie connessioni, sia in grado di operare non considerando nel processo i layer intermedi. Una domanda che ci si potrebbe porre è la seguente: "a cosa serve saltare dei layer nell'esecuzione? non è controproducente?". Questo tipo di domanda è a mio avviso, molto opportuna, in quanto è vero che la logica generale sia contro intuitiva, allo stesso tempo però si può osservare che, lo skip di layer:
\begin{itemize}
	\item risolve il noto ed importante problema dei \textbf{vanishing gradients}, ottimizzando la rete neurale. Nell'apprendimento automatico, il problema del Vanishing gradient si incontra quando si addestrano reti neurali con metodi di apprendimento basati sul gradiente e sulla back-propagation. In tali metodi, durante ogni iterazione dell'addestramento, ciascuno dei pesi della rete neurale riceve un aggiornamento proporzionale alla derivata parziale della funzione di errore rispetto al peso corrente. Il problema è che in alcuni casi il gradiente sarà di dimensioni infinitesimali, impedendo al peso di cambiare valore in maniera ottimale;
	\item \textbf{mitiga il problema di degradazione, meglio noto come accuracy problem (problema di precisione)}. La degradazione si verifica quando si ha a che fare con una rete con un grande numero di layer, perché è formalmente dimostrato che, aumentare il numero di strati, comporta un margine di errore più ampio. 
\end{itemize} 
Nel trasformer, l'importanza delle connessioni residue si può percepire dopo aver studiato ed analizzato con precisione tutti i singoli passaggi, indicati nel prossimo paragrafo.
 \paragraph{Si parte} unendo l'output generato dal \textbf{multi-level attention layer} all'input originale. Il risultato ottenuto sarà poi normalizzato, passando attraverso il normalization layer. Supponendo che $multiLevelHeadOutput$ sia l'output ottenuto dal multi-headed layer e che $input$ sia l'input originale,dato in pasto allo stesso, il tutto può essere formalizzato attraverso la formula:
\begin{itemize}
	\item $ LayerNorm(ResidualConnection = multiLevelHeadOutput, input)$
\end{itemize}

Ciò che ottengo dalla normalizzazione, passa attraverso una rete neurale feed-forward, che dunque non ammette backtracking e non ha memoria, nemmeno a breve termine. 
La rete neurale artificiale citata, è composta da 2 layer lineari, come quelli visti in precedenza, con la differenza che fra essi, viene collocato un layer ReLu.
La funzione ReLu è una funzione di attivazione non lineare, molto nota in ambito deep learning. Si tratta di una sigla che sta per Rectified Linear Unit, semplificabile in "rettificatore". La particolarità della funzione è che attiva \textbf{solo un} neurone alla volta e non gli è concesso attivarli tutti contemporaneamente. 

\paragraph{Il valore di uscita} dalla rete feed forward, verrà a sua volta legato all'input originale, formando un'ulteriore connessione residua, che verrà poi normalizzata da un layer di normalizzazione. Incorporare al modello una rete feed-forward serve per elaborare ulteriormente l'output della attention, dandogli potenzialmente una rappresentazione più ricca.
\paragraph{Questo ultimo tassello} conclude la logica procedurale del modello encoder, il quale scopo, come già spiegato nell'introduzione del capitolo, è di offrire al decoder delle indicazioni da seguire, circa le parole alla quale deve essere dedicata una maggiore attenzione. Come con l'attention è possibile creare uno stack di differenti encoder, dove ogni singolo elemento della pila ha possibilità di imparare diverse rappresentazioni dell'attention. Quest'ultimo aspetto è molto interessante, perché aumenta il potere predittivo del rete del modello.

\subsubsection{Decoder}

Il ruolo del decoder consiste nel generare sequenze di testo, la sua struttura è molto simile a quella dell'encoder. 
In particolare, ha a sua disposizione:
\begin{itemize}
	\item 2 multi-headed attention layers;
	\item un feed fordward layer, ovvero una rete neurale senza backtracking;
	\item un sub-layer di normalizzazione per ogni layer padre presente, oltre alla presenza di residual connections derivanti dall'utilizzo di una residual neural network (rnn);
\end{itemize}

Si tratta a tutti gli effetti di un modello autoregressivo (in inglese autoregressive model). 
Il decoder infatti, prende in ingresso gli output generati dall'encoder, oltre alle informazioni sull'attention degli stessi. Il layer elabora richieste fino a quando non viene generato l'<end> token, il quale indica la fine della frase.

\subsubsection{Modello autoregressivo}

Un modello autoregressivo si fonda sulla misurazione della correlazione fra osservazioni, effettuate ad un istante temporale antecedente rispetto al tempo di misurazione, allo scopo di predirre il valore che la variabile considerata assumerà al prossimo step (il nostro output). Nel caso in cui la variabile di input e output varino nella stessa direzione, ovvero incrementando o diminuendo insieme, si otterrà una correlazione positiva. Al contrario, se la variazione segue il senso opposto, la correlazione sarà negativa. All'aumentare del valore di correlazione, che sia negativo o positivo, aumenta la probabilità che attraverso l'input la macchina sia in grado di prevedere il futuro. Dal momento che la correlazione è fra la variabile al presente e sè stessa al passo precedente, si parla di autocorrelazione.

\subsubsection{Passaggi effettuati nella fase di decoding}

\subsubsection{Embedding dell'output e positional encoding}

L'input passa attraverso due layer, si tratta dell'embedding layer e il positional encoding layer. Il procedimento denota la posizione delle parole analizzate rispetto alla frase presa come riferimento.

\subsubsection{Decoder multi headed attention 1}

L'output ottenuto dal passaggio precedente vengono dati in pasto al primo dei due multi headed attention layer. In questa fase, il componente ha lo scopo di estrarre i punteggi relativi all'attention dell'input.
Non bisogna confonderlo con il multi-headed attention layer presente nella struttura dell'encoder, questo perché, l'autoregressione richiede soluzioni leggermente diverse. In particolare, dal momento che processa l'input parola per parola è ritenuto necessario evitare che token futuri vengano condizionati. Ad esempio, supponendo che il modello stia analizzando la parola "tutto", della frase $\{$ "Io", "tutto", "bene"; "tu"?$\}$, esso non deve avere accesso alle parole successive, perché generate in un istante futuro. Detto ciò, quindi, è importante fare in modo che la computazione di una parola, possa essere condizionata solo dalle precedenti (generate anteriormente), non dalle successive (generate successivamente).
Il metodo che rende possibile tutto ciò prende il nome di masking, e segue il seguente funzionamento:
\begin{itemize}
	\item per evitare che il decoder prenda in considerazione token futuri, viene applicata una "look ahead mask" prima di calcolare il softmax della matrice dei valori e dopo la normalizzazione degli score (osservare la sezione della attention per chiarire). La maschera è una matrice delle stesse dimensioni della sealed matrix, ovvero la matrice degli score in seguito alla normalizzazione. Le celle sopra la diagonale assumono valore - $\infty$, mentre le altre valore 0. Il risultato è la matrice chiamata \textbf{masked scores}. 
	\item il motivo per il quale si applica questo tipo di maschera, è legato al fatto che, applicando il softmax (visto nei capitoli precedenti), i valori con -$\infty$ verranno sostituiti da zeri. La matrice risultante valorizza il punteggio di attention che verrà attribuito ad ogni token, a questo proposito, abbiamo che azzerando la parte sopra alla diagonale della matrice, stiamo implicitamente escludendo i token futuri dal calcolo. 
	\\
	\begin{table}[h]                        %ambiente tabella
		%(serve per avere la legenda)                          %centra nella pagina la tabella
			\begin{tabular}{r|c|c}                  %tre colonne con righe verticali
				%   prodotte con |
				\hline \hline                           %inserisce due righe orizzontali
				$0.7$ & * & *\\           %& separa le colonne e con
				\hline                                  %inserisce una riga orizzontale
				$0.5$ & $0.1$ & *\\           %  \\ va a capo
				\hline                                  %inserisce una riga orizzontale
				$0.3$ & $0.2$ & $0.2$\\
				\hline \hline                           %inserisce due righe orizzontali
			\end{tabular}
		+
					\begin{tabular}{r|c|c}                  %tre colonne con righe verticali
			%   prodotte con |
			\hline \hline                           %inserisce due righe orizzontali
			$0$ & $-\infty$ & $ - \infty$\\           %& separa le colonne e con
			\hline                                  %inserisce una riga orizzontale
			$0$ & $0$ & $-\infty$\\           %  \\ va a capo
			\hline                                  %inserisce una riga orizzontale
			$0$ & $0$ & $0$\\
			\hline \hline                           %inserisce due righe orizzontali
		\end{tabular}
		=
				\begin{tabular}{r|c|c}                  %tre colonne con righe verticali
		%   prodotte con |
		\hline \hline                           %inserisce due righe orizzontali
		$0.7$ & $-\infty$ & $ - \infty$\\           %& separa le colonne e con
		\hline                                  %inserisce una riga orizzontale
		$0.5$ & $0.1$ & $-\infty$\\           %  \\ va a capo
		\hline                                  %inserisce una riga orizzontale
		$0.3$ & $0.2$ & $0.2$\\
		\hline \hline                           %inserisce due righe orizzontali
	\end{tabular}
			\caption[legenda elenco tabelle]{L'esempio sovrastante ci mostra l'applicazione della maschera alla matrice degli score (post normalizzazione). In particolare, alla matrice degli score viene sommata la maschera (matrice centrale), che va ad incidere solo sui valori situati sopra la diagonale. Gli asterischi presenti nella matrice di sinistra, vanno ad indicare valori randomici, non è importante conoscerli, dal momento che diventeranno pari a - $\infty$}.
	\end{table}
	
		\begin{table}[h]                        %ambiente tabella
		%(serve per avere la legenda)                          %centra nella pagina la tabella
		$softmax($
		\begin{tabular}{r|c|c}                  %tre colonne con righe verticali
			%   prodotte con |
			\hline \hline                           %inserisce due righe orizzontali
			$0.7$ & * & *\\           %& separa le colonne e con
			\hline                                  %inserisce una riga orizzontale
			$0.5$ & $0.1$ & *\\           %  \\ va a capo
			\hline                                  %inserisce una riga orizzontale
			$0.3$ & $0.2$ & $0.2$\\
			\hline \hline                           %inserisce due righe orizzontali
		\end{tabular}
		$)$
		= 
				\begin{tabular}{r|c|c}                  %tre colonne con righe verticali
			%   prodotte con |
			\hline \hline                           %inserisce due righe orizzontali
			$0.7$ & 0 & 0\\           %& separa le colonne e con
			\hline                                  %inserisce una riga orizzontale
			$0.5$ & $0.1$ & 0\\           %  \\ va a capo
			\hline                                  %inserisce una riga orizzontale
			$0.3$ & $0.2$ & $0.2$\\
			\hline \hline                           %inserisce due righe orizzontali
		\end{tabular}
		\caption[legenda elenco tabelle]{Applicazione della funzione softmax alla matrice ottenuta in seguito al masking}.
			\end{table}
		\subsubsection{Multi headed attention layer 2}
		Riceve in input il vettori delle query e values derivanti dall'encoder, mentre il vettore degli scores è l'output ottenuto dal primo layer di multi head attention. 
		
		\subsubsection{Feed forward linear layer (Classifier)}
		
		L'output ottenuto in seguito all'elaborazione delle informazioni a carico del secondo layer di attention, passa attraverso una rete feed forward, la quale accede ad un classifier.
		A classifier in machine learning is an algorithm that automatically orders or categorizes data into one or more of a set of “classes.” One of the most common examples is an email classifier that scans emails to filter them by class label: Spam or Not Spam.
		Il classifier (o classificatore, in italiano), in questo caso, crea una classe per ogni singola parola data in input, quindi, per chiarire il concetto, data in input una frase composta da 500 parole, il classificatore la mapperà in 500 diverse classi, indicizzate e con identificatore univoco. L'output del classificatore viene poi immesso nel softmax layer, il quale ha prodotto i punteggi di probabilità compresi tra zero e uno per ciascuna classe.  Infine il modello prenderà la classe con il punteggio di predittività più alto, restituendolo in output, si tratta del risultato della predizione del modello.  La parola risultante verrà poi aggiunta all'elenco degli input del decodificatore, il quale continuerà di nuovo la decodifica fino a quando non incontrerà il token di fine, la quale classe ha la previsione di probabilità più alta. Il decoder, similmente all'encoder, può essere impilato e sovrapposto, ogni strato riceve l'input dall'encoder e dai livelli precedenti. Impilando i livelli, il modello può imparare a estrarre e concentrarsi su diverse combinazioni attraverso l'attention, aumentando potenzialmente il suo potere predittivo.



\end{itemize}

\section{Seconda Sezione}
Ora vediamo un elenco puntato:
\begin{itemize}                         %crea un elenco puntato
\item primo oggetto
\item secondo oggetto
\end{itemize}

\section{Altra Sezione}
Vediamo un elenco descrittivo:
\begin{description}                     %crea un elenco descrittivo
  \item[OGGETTO1] prima descrizione;
  \item[OGGETTO2] seconda descrizione;
  \item[OGGETTO3] terza descrizione.
\end{description}

%%%%%%%%%%%%%%%%%%%%%%%%%%%%%%%%%%%%%%%%%crea una sottosezione
\subsection{Altra SottoSezione}
%%%%%%%%%%%%%%%%%%%%%%%%%%%%%%%%%%%%%%%%%crea una sottosottosezione
\subsubsection{SottoSottoSezione}Questa sottosottosezione non viene
numerata, ma \`e solo scritta in grassetto.
\section{Altra Sezione}                 %crea una sottosezione
Vediamo la creazione di una tabella; la tabella \ref{tab:uno}
(richiamo il nome della tabella utilizzando la label che ho messo sotto):
la facciamo di tre righe e tre colonne, la prima colonna
``incolonnata'' a destra (r) e le altre centrate (c):\\
\begin{table}[h]                        %ambiente tabella
                                        %(serve per avere la legenda)
\begin{center}                          %centra nella pagina la tabella
\begin{tabular}{r|c|c}                  %tre colonne con righe verticali
                                        %   prodotte con |
\hline \hline                           %inserisce due righe orizzontali
$(1,1)$ & $(1,2)$ & $(1,3)$\\           %& separa le colonne e con
\hline                                  %inserisce una riga orizzontale
$(2,1)$ & $(2,2)$ & $(2,3)$\\           %  \\ va a capo
\hline                                  %inserisce una riga orizzontale
$(3,1)$ & $(3,2)$ & $(3,3)$\\
\hline \hline                           %inserisce due righe orizzontali
\end{tabular}
\caption[legenda elenco tabelle]{legenda tabella}\label{tab:uno}
\end{center}
\end{table}
\section{Altra Sezione}\label{sec:prova}%posso mettere le label anche
                                        %   alle section
\subsection{Listati dei programmi}
\subsubsection{Primo Listato}
\begin{verbatim}
        In questo ambiente     posso scrivere      come voglio,
lasciare gli spazi che voglio e non % commentare quando voglio
e ci sarà scritto tutto.
Quando lo uso è meglio che disattivi il Wrap del WinEdt
\end{verbatim}
	\clearpage{\pagestyle{empty}\cleardoublepage}
\chapter{Secondo capitolo}                %crea il capitolo
%%%%%%%%%%%%%%%%%%%%%%%%%%%%%%%%%%%%%%%%%imposta l'intestazione di pagina

Questo \`e il secondo capitolo.
\section{Prima Sezione}                 %crea la sezione
Questa \`e la prima sezione.

\section{Seconda Sezione}                 %crea la sezione
Questa \`e la seconda sezione.

	\clearpage{\pagestyle{empty}\cleardoublepage}
\chapter{Terzo capitolo}                %crea il capitolo
%%%%%%%%%%%%%%%%%%%%%%%%%%%%%%%%%%%%%%%%%imposta l'intestazione di pagina

Questo \`e il terzo capitolo.

\section{Prima Sezione}                 %crea la sezione
Questa \`e la prima sezione.

\newpage

\section{Seconda Sezione}                 %crea la sezione
Questa \`e la seconda sezione.

\newpage

\section{Terza Sezione}                 %crea la sezione
Questa \`e la terza sezione.

	

	%%%%%%%%%%%%%%%%%%%%%%%%%
	% inizio parte finale del documento
	%
	% eventuali appendici, bibliografia obbligatoria,
	% eventuale lista delle tabelle e delle figure (nel caso decommentare 
	% la riga con i comandi \listoffigures e \listoftables)
	%%%%%%%%%%%%%%%%%%%%%%%%%
	
    %%%%%%%%%%%%%%%%%%%%%%%%%%%%%%%%%%%%%%%%%non numera l'ultima pagina sinistra
\clearpage{\pagestyle{empty}\cleardoublepage}
%%%%%%%%%%%%%%%%%%%%%%%%%%%%%%%%%%%%%%%%%per fare le conclusioni
\chapter*{Conclusioni}
%%%%%%%%%%%%%%%%%%%%%%%%%%%%%%%%%%%%%%%%%imposta l'intestazione di pagina
\rhead[\fancyplain{}{\bfseries
CONCLUSIONI}]{\fancyplain{}{\bfseries\thepage}}
\lhead[\fancyplain{}{\bfseries\thepage}]{\fancyplain{}{\bfseries
CONCLUSIONI}}
%%%%%%%%%%%%%%%%%%%%%%%%%%%%%%%%%%%%%%%%%aggiunge la voce Conclusioni
                                        %   nell'indice
\addcontentsline{toc}{chapter}{Conclusioni} Queste sono le
conclusioni.\\

Lorem ipsum dolor sit amet, consectetur adipiscing elit. Quisque a magna quis nunc venenatis vestibulum. Curabitur commodo efficitur ipsum, non ullamcorper tellus. Duis dictum commodo nisi nec venenatis. Donec euismod pulvinar finibus. Suspendisse lorem mi, suscipit quis faucibus ut, luctus in justo. Cras pulvinar arcu ut ullamcorper pulvinar. Aliquam dictum tortor quis diam luctus, quis tristique tortor ultrices. Integer et lacus a velit efficitur convallis. Morbi enim erat, fermentum vel nulla id, viverra vehicula nisi. Integer non auctor leo, eu convallis massa. Cras eu cursus ligula. Nunc non purus et sem vehicula viverra ut nec nibh.

Quisque posuere purus quis eros auctor efficitur. Etiam mattis vitae nulla et blandit. Nulla a orci magna. Cras ac elit enim. Vestibulum nec nisl metus. Mauris congue velit nec malesuada scelerisque. Sed dignissim, enim vitae semper fermentum, mauris leo vestibulum nisl, in malesuada nibh felis nec dui.

Nullam sit amet tellus eget mi varius commodo. Vestibulum sit amet egestas odio. Nam in ullamcorper quam, nec efficitur augue. Curabitur eget elit in leo eleifend tempor vel lobortis lorem. Duis neque dui, tempus eu sollicitudin ac, lobortis sit amet odio. Morbi eleifend, tellus a varius consequat, enim erat sagittis justo, ac rutrum ipsum augue in leo. Suspendisse non mi ante.

Praesent sed pretium dui, id volutpat tortor. Suspendisse tortor lorem, vestibulum vitae ullamcorper vitae, tincidunt nec leo. Proin interdum congue blandit. Ut bibendum sagittis leo, nec venenatis urna mollis id. Donec nec erat non justo maximus venenatis. In mollis elit eu odio maximus porta. Vestibulum varius turpis sit amet orci blandit, vitae volutpat erat viverra.

Suspendisse nunc urna, elementum ut purus a, sagittis porta velit. Integer ultricies convallis tortor id pellentesque. Duis et sem a mi bibendum congue. Morbi ut tellus cursus, laoreet ipsum rutrum, condimentum felis. Proin velit mi, ultricies a urna nec, facilisis pretium mi. Pellentesque tristique interdum purus, a facilisis mi tempor quis. Sed finibus venenatis ligula porttitor porttitor. Suspendisse cursus lorem nec velit commodo fringilla.
Lorem ipsum dolor sit amet, consectetur adipiscing elit. Quisque a magna quis nunc venenatis vestibulum. Curabitur commodo efficitur ipsum, non ullamcorper tellus. Duis dictum commodo nisi nec venenatis. Donec euismod pulvinar finibus. Suspendisse lorem mi, suscipit quis faucibus ut, luctus in justo. Cras pulvinar arcu ut ullamcorper pulvinar. Aliquam dictum tortor quis diam luctus, quis tristique tortor ultrices. Integer et lacus a velit efficitur convallis. Morbi enim erat, fermentum vel nulla id, viverra vehicula nisi. Integer non auctor leo, eu convallis massa. Cras eu cursus ligula. Nunc non purus et sem vehicula viverra ut nec nibh.

Quisque posuere purus quis eros auctor efficitur. Etiam mattis vitae nulla et blandit. Nulla a orci magna. Cras ac elit enim. Vestibulum nec nisl metus. Mauris congue velit nec malesuada scelerisque. Sed dignissim, enim vitae semper fermentum, mauris leo vestibulum nisl, in malesuada nibh felis nec dui.

Nullam sit amet tellus eget mi varius commodo. Vestibulum sit amet egestas odio. Nam in ullamcorper quam, nec efficitur augue. Curabitur eget elit in leo eleifend tempor vel lobortis lorem. Duis neque dui, tempus eu sollicitudin ac, lobortis sit amet odio. Morbi eleifend, tellus a varius consequat, enim erat sagittis justo, ac rutrum ipsum augue in leo. Suspendisse non mi ante.

Praesent sed pretium dui, id volutpat tortor. Suspendisse tortor lorem, vestibulum vitae ullamcorper vitae, tincidunt nec leo. Proin interdum congue blandit. Ut bibendum sagittis leo, nec venenatis urna mollis id. Donec nec erat non justo maximus venenatis. In mollis elit eu odio maximus porta. Vestibulum varius turpis sit amet orci blandit, vitae volutpat erat viverra.

Suspendisse nunc urna, elementum ut purus a, sagittis porta velit. Integer ultricies convallis tortor id pellentesque. Duis et sem a mi bibendum congue. Morbi ut tellus cursus, laoreet ipsum rutrum, condimentum felis. Proin velit mi, ultricies a urna nec, facilisis pretium mi. Pellentesque tristique interdum purus, a facilisis mi tempor quis. Sed finibus venenatis ligula porttitor porttitor. Suspendisse cursus lorem nec velit commodo fringilla.
    %\clearpage{\pagestyle{empty}\cleardoublepage}

\cleardoublepage

\rhead[\fancyplain{}{\bfseries BIBLIOGRAFIA}]{\fancyplain{}{\bfseries\thepage}}
\lhead[\fancyplain{}{\bfseries\thepage}]{\fancyplain{}{\bfseries BIBLIOGRAFIA}}
%%%%%%%%%%%%%%%%%%%%%%%%%%%%%%%%%%%%%%%%% aggiunge l'intestazione di pagina
%\chapter*{Bibliografia}


\phantomsection


\addcontentsline{toc}{chapter}{Bibliografia}
%%%%%%%%%%%%%%%%%%%%%%%%%%%%%%%%%%%%%%%%% aggiunge la voce Bibliografia nell'indice





\bibliography{backMatter/biblio}{}
\bibliographystyle{plain}


    \rhead[\fancyplain{}{\bfseries \leftmark}]{\fancyplain{}{\bfseries
\thepage}}
%%%%%%%%%%%%%%%%%%%%%%%%%%%%%%%%%%%%%%%%%aggiunge la voce Bibliografia
                                        %   nell'indice

%%%%%%%%%%%%%%%%%%%%%%%%%%%%%%%%%%%%%%%%%non numera l'ultima pagina sinistra
\clearpage{\pagestyle{empty}\cleardoublepage}
\chapter*{Ringraziamenti}
\thispagestyle{empty}

\addcontentsline{toc}{chapter}{Ringraziamenti}

Qui possiamo ringraziare il mondo intero!!!!!!!!!!\\
Ovviamente solo se uno vuole, non \`e obbligatorio.
	
		
	%\input{./Appendice/appendice.tex}
	%%\clearpage{\pagestyle{empty}\cleardoublepage}

\cleardoublepage

\rhead[\fancyplain{}{\bfseries BIBLIOGRAFIA}]{\fancyplain{}{\bfseries\thepage}}
\lhead[\fancyplain{}{\bfseries\thepage}]{\fancyplain{}{\bfseries BIBLIOGRAFIA}}
%%%%%%%%%%%%%%%%%%%%%%%%%%%%%%%%%%%%%%%%% aggiunge l'intestazione di pagina
%\chapter*{Bibliografia}


\phantomsection


\addcontentsline{toc}{chapter}{Bibliografia}
%%%%%%%%%%%%%%%%%%%%%%%%%%%%%%%%%%%%%%%%% aggiunge la voce Bibliografia nell'indice





\bibliography{backMatter/biblio}{}
\bibliographystyle{plain}


	
	\nocite{*}

	%\cleardoublepage
	%\addcontentsline{toc}{chapter}{Bibliografia}


	
	%\listoffigures
	%\listoftables


\end{document}
